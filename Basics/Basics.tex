\documentclass[11pt]{beamer}  %% versione proiettore
%%\documentclass[11pt,handout]{beamer} %% versione stampa
\usepackage{lucidiJb-2ed}

\usepackage{relsize}

\mode<article>
{
  \usepackage{fullpage}
  \usepackage{hyperref}
}

\mode<presentation>
{
  \setbeamertemplate{background canvas}[vertical shading][bottom=red!10,top=blue!10]
  \usetheme{Ethereum}
  \usefonttheme[onlysmall]{structurebold}
}

\subtitle{Learning Ethereum}
\title{The Basics}
\institute{Universit\`a di Verona, Italy}
\date{January 2020}

\setbeamercovered{invisible}

\def\codesize{\smaller}
\def\<#1>{\codeid{#1}}
\newcommand{\codeid}[1]{\ifmmode{\mbox{\codesize\ttfamily{#1}}}\else{\codesize\ttfamily #1}\fi}

\begin{document}

\begin{frame}
  \titlepage
\end{frame}

\begin{frame}
  \frametitle{What is Ethereum?}

  \begin{greenbox}{}
    An open source, globally decentralized computing infrastructure
    that executes programs called \emph{smart contracts}, written
    in a Turing-complete programming language, translated into
    bytecode and run on a virtual machine. It uses a
    blockchain to synchronize and store the system's state changes
    (key/value tuples), along
    with a cryptocurrency called \emph{ether} to meter and constrain
    execution resource costs. It enables developers to build
    decentralized applications with built-in economic functions.
  \end{greenbox}

  \bigskip
  
  \emph{gas} is bought when a transaction starts, at a maximal gas price.
  
\end{frame}

\begin{frame}\frametitle{DApps}

  \begin{greenbox}{DApps are smart contracts + web frontend (web3)}
    \begin{center}
      \includegraphics[width=\textwidth,clip=false]{pictures/dapps.png}
    \end{center}
  \end{greenbox}

\end{frame}

\begin{frame}\frametitle{People behind Ethereum}

  \begin{center}
  \begin{tabular}{c@{\hskip 1.5cm}c}
    \includegraphics[scale=.3,clip=false]{pictures/vitalik_buterin.jpg} &
    \includegraphics[scale=.247,clip=false]{pictures/gavin_wood.jpg} \\
    Vitalik Buterin & Gavin Wood
  \end{tabular}
  \end{center}
  
\end{frame}

\begin{frame}\frametitle{References used in this course}

  \begin{center}
  \begin{tabular}{c@{\hskip 1.5cm}c}
    \includegraphics[scale=.3,clip=false]{pictures/mastering-ethereum.jpg} &
    \includegraphics[scale=.3,clip=false]{pictures/building-games.jpg}
  \end{tabular}
  \end{center}

  The first: \url{https://github.com/ethereumbook/ethereumbook}

\end{frame}

\begin{frame}\frametitle{What is inside the blocks?}

  \begin{center}
    \includegraphics[width=\textwidth,clip=false]{pictures/inside-blocks.jpg}
  \end{center}

\end{frame}

\begin{frame}\frametitle{Accounts, state and receipts are \alert{not} in blockchain}

  They are kept inside a database, local to each node.

  \begin{center}
    \includegraphics[scale=0.38,clip=false]{pictures/accounts-state-receipts.png}
  \end{center}

\end{frame}

\begin{frame}\frametitle{Mining with proof of work (PoW)}
  Each miner works as follows:
  \begin{enumerate}
  \item collect transactions received from users or from other nodes
  \item choose and execute some of these transactions: the choice is free,
    as long as choice and execution respect the \alert{consensus rules}
  \item build a candidate block containing (among other):
    \begin{itemize}
    \item hash of the previous block
    \item selected transactions
    \item hash of the transaction receipts
    \item hash of the state at the end of the last transaction in the block
    \item address of the beneficiary miner (itself)
    \item timestamp
    \item the current level of difficulty $d$
    \item a random \alert{nonce}    
    \end{itemize}
  \item hash the candidate block
    \begin{itemize}
    \item if this hash starts with at least $d$ zero's, add the candidate block to
      your local copy of the blockchain and
      spread it among peers
    \item otherwise, go back to step 3, but change nonce
    \end{itemize}
  \end{enumerate}
\end{frame}

\begin{frame}\frametitle{Multiple chains, only one emerging}

  \begin{center}
    \includegraphics[scale=0.29,clip=false]{pictures/chains.png}
  \end{center}

\end{frame}

\begin{frame}\frametitle{Changing the story with the 51\% attack}

  \begin{center}
    \includegraphics[width=\textwidth,clip=false]{pictures/51-a.png}
  \end{center}

\end{frame}

\begin{frame}\frametitle{Changing the story with the 51\% attack}

  \begin{center}
    \includegraphics[width=\textwidth,clip=false]{pictures/51-b.png}
  \end{center}

\end{frame}

\begin{frame}\frametitle{Ether (ETH) price over the years}

  \begin{center}
    \includegraphics[scale=0.29,clip=false]{pictures/ethereum.png}
  \end{center}

\end{frame}

\begin{frame}\frametitle{Ethereum Classic Ether (ETC) price over the years}

  \begin{center}
    \includegraphics[scale=0.29,clip=false]{pictures/ethereum-classic.png}
  \end{center}

\end{frame}

\begin{frame}\frametitle{Ether denominations and unit names}

  \begin{center}
    \includegraphics[scale=0.2,clip=false]{pictures/ether-denominations.png}
  \end{center}

\end{frame}

\begin{frame}\frametitle{Wallets}

  \begin{greenbox}{}
    A wallet is a
    software application that helps you manage your Ethereum account(s),
    by keeping your keys, creating and broadcasting transactions:
    \begin{itemize}
    \item mobile (Jaxx)
    \item desktop (Jaxx, Emerald Wallet)
    \item web-based (MetaMask, MyEtherWallet)
    \end{itemize}
  \end{greenbox}

\end{frame}

\begin{frame}\frametitle{MetaMask: create account}

  \begin{center}
    \includegraphics[width=\textwidth,clip=false]{pictures/metamask-account.png}
  \end{center}

\end{frame}

\begin{frame}\frametitle{MetaMask: connect to the faucet}

  \begin{center}
    \includegraphics[scale=0.3,clip=false]{pictures/metamask-faucet.png}
  \end{center}

\end{frame}

\begin{frame}\frametitle{MetaMask: transaction for getting 1 ETH}

  \begin{center}
    \includegraphics[width=\textwidth,clip=false]{pictures/metamask-faucet-transaction.png}
  \end{center}

\end{frame}

\begin{frame}\frametitle{MetaMask: detail of the transaction}

  \begin{center}
    \includegraphics[scale=0.4,clip=false]{pictures/metamask-faucet-transaction-details.png}
  \end{center}

\end{frame}

\begin{frame}\frametitle{MetaMask: transactions of our account}

  \begin{center}
    \includegraphics[width=\textwidth,clip=false]{pictures/metamask-faucet-transactions.png}
  \end{center}

\end{frame}

\begin{frame}\frametitle{Externally owned accounts (EOA) and contracts}

  \begin{greenbox}{}
    EOAs have keys, contracts have code. Both have an address.
  \end{greenbox}
  
  \begin{center}
    \includegraphics[scale=0.35,clip=false]{pictures/eoa-contract.jpg}
  \end{center}

\end{frame}

\begin{frame}\frametitle{Remix: web IDE for Solidity}

  \begin{greenbox}{}
    \url{https://remix.ethereum.org}
  \end{greenbox}

  \begin{center}
    \includegraphics[width=\textwidth,clip=false]{pictures/faucet_sol.png}
  \end{center}

\end{frame}

\begin{frame}\frametitle{Remix: deploy an instance of the contract}

  \begin{greenbox}{}
    Remix will create for us a transaction with the \<0x0> address as destination.
  \end{greenbox}

  \begin{center}
    \includegraphics[scale=0.45,clip=false]{pictures/deploy-faucet.png}
  \end{center}

\end{frame}

\begin{frame}\frametitle{Remix: an instance of \<Faucet> has been deployed}

  \begin{center}
    \includegraphics[scale=0.5,clip=false]{pictures/deployed-faucet.png}
  \end{center}

\end{frame}

\begin{frame}\frametitle{Check the outcome on Etherscan}

  \begin{greenbox}{}
    \url{https://ropsten.etherscan.io}
  \end{greenbox}

  \begin{center}
    \includegraphics[width=\textwidth,clip=false]{pictures/faucet-etherscan.png}
  \end{center}

\end{frame}

\begin{frame}\frametitle{Send one Ether to the \<Faucet> through MetaMask}

  This corresponds to calling the default payable function.
  
  \begin{center}
    \includegraphics[width=\textwidth,clip=false]{pictures/faucet-etherscan2.png}
  \end{center}

\end{frame}

\begin{frame}\frametitle{Run the \<withdraw> function of the \<Faucet> in Remix}

  \begin{center}
    \includegraphics[width=\textwidth,clip=false]{pictures/faucet-withdraw.png}
  \end{center}

  \begin{center}
    $17$ zero's
  \end{center}
  
\end{frame}

\begin{frame}\frametitle{Check again on Etherscan}

  The faucet worked properly!
  
  \begin{center}
    \includegraphics[width=\textwidth,clip=false]{pictures/faucet-etherscan3.png}
  \end{center}

\end{frame}

\begin{frame}\frametitle{The \<Faucet> has originated an internal transaction}

  \begin{center}
    \includegraphics[width=\textwidth,clip=false]{pictures/faucet-etherscan-internal.png}
  \end{center}

  \begin{center}
    \<msg.sender.transfer(withdraw\_amount)>
  \end{center}

\end{frame}

\begin{frame}\frametitle{Ethereum clients}

  \begin{greenbox}{}
    An Ethereum client is a software application that implements
    the Ethereum specification (the \emph{Yellow Paper}) and communicates
    over the peer-to-peer network with other Ethereum clients.
  \end{greenbox}

  \medskip

  \begin{itemize}
  \item Parity (Rust)
  \item Geth (Go)
  \item \<cpp-ethereum> (C++)
  \item \<pyethereum> (Python)
  \item Mantis (Scala)
  \item Harmony (Java)
  \end{itemize}
  
\end{frame}

\begin{frame}\frametitle{Client types}

  \begin{greenbox}{Full nodes}
    A full node stores the whole blockchain ($\sim$200GB) and
    takes part in mining. Very slow to synchronize.
    It can query the blockchain offline and
    without letting a third party know the information that it's reading.
  \end{greenbox}

  \bigskip

  \begin{greenbox}{Remote clients}
    A remote client does not store the blockchain but
    depends on another full node for operation. They are wallets
    that create and broadcast transactions (for instance, MetaMask).
  \end{greenbox}

  \bigskip

  \begin{greenbox}{Networks}
    The \emph{mainnet}, or a \emph{testnet}, or a local blockchain
    (Ganache).
  \end{greenbox}
  
\end{frame}

\begin{frame}\frametitle{Infura: a cloud service to a full node}

  \url{https://infura.io}

  \begin{center}
    \includegraphics[width=\textwidth,clip=false]{pictures/infura.png}
  \end{center}

\end{frame}

\begin{frame}[fragile]\frametitle{Infura: ask for client version}

  \begin{greenbox}{}
    Register to Infura and create a project. Use the provided
    ID to build a query to Infura's JSON-RPC API:

\begin{semiverbatim}
$ \alert{curl https://mainnet.infura.io/v3/05....4
  -X POST
  -H "Content-Type: application/json"
  -d '\{"jsonrpc":"2.0","method":"web3_clientVersion",
  "params": [],"id":1\}'}
{\color{blue}\{"jsonrpc":"2.0","id":1,
 "result":"Geth/v1.9.9-omnibus-e320ae4c-20191206
               /linux-amd64/go1.13.4"\}}
\end{semiverbatim}

\end{greenbox}

\end{frame}

\begin{frame}[fragile]\frametitle{Infura: ask for gas price}

  \begin{greenbox}{}
    Use your Infura ID to build a query to Infura's JSON-RPC API:

\begin{semiverbatim}
$ \alert{curl https://mainnet.infura.io/v3/05....4
  -X POST
  -H "Content-Type: application/json"
  -d '\{"jsonrpc":"2.0","method":"eth_gasPrice",
       "params": [],"id":1\}'}
{\color{blue}\{"jsonrpc":"2.0","id":1,"result":"0x3b9aca00"\}}
$ \alert{echo $((0x3b9aca00))}
{\color{blue}1000000000}
\end{semiverbatim}

  \end{greenbox}

\end{frame}

\begin{frame}\frametitle{Infura implements Ethereum JSON-RPC API}

  \begin{center}
    \includegraphics[scale=0.47,clip=false]{pictures/infura-json-rpc.png}
  \end{center}

\end{frame}

\begin{frame}[fragile]\frametitle{A Java client that connects to Infura}

  \begin{greenbox}{Use Eclipse to create a new Java Maven project}
    Add the following dependency to \<pom.xml>:
\begin{verbatim}
<project xmlns="http://maven.apache.org/POM/4.0.0"...>
  <modelVersion>4.0.0</modelVersion>
  <groupId>it.univr</groupId>
  <artifactId>ethereum</artifactId>
  <version>0.0.1-SNAPSHOT</version>
  <name>LearningEthereum</name>
  <dependencies>
    <dependency>
      <groupId>org.web3j</groupId>
      <artifactId>core</artifactId>
      <version>4.5.11</version>
    </dependency>
  </dependencies>
</project>
\end{verbatim}
  \end{greenbox}
\end{frame}

\begin{frame}\frametitle{A Java client that connects to Infura}

  \begin{center}
    \includegraphics[width=\textwidth,clip=false]{pictures/client-version-java.png}
  \end{center}

  \bigskip

  \url{https://javadoc.io/doc/org.web3j}

\end{frame}

\begin{frame}\frametitle{Mobile remote client: Jaxx}

  \begin{center}
    \includegraphics[scale=0.1,clip=false]{pictures/jaxx.png}
  \end{center}

\end{frame}

\end{document}
